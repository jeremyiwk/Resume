
\documentclass[]{letter}
\usepackage[none]{hyphenat}
\usepackage[hmarginratio=1:1]{geometry}
\begin{document}

%{\Large Cover Letter}\\
%\end{center}
%\vspace{0.2 cm}

Dear Flow Science Hiring Team, \\

I am very excited about jobs at Flow-3D because I have a deep personal interest in computational physics and simulations, and my primary professional goal is to find a job working in this field. In particular, I am interested in finding a career building software to implement numerical solutions to partial differential equations for simulating physical systems. My current experience with computational physics lies in two primary domains: molecular simulations of complex fluids, and simulations of charged particle optical systems. \\

As a graduate student at the University of Oregon, my research focused on using computational methods to build models of individual macromolecules or ensembles of macromolecules using a Langevin equation or the Ornstein–Zernike equation, respectively. In general, this required developing a theoretical model and using data from molecular dynamics (MD) simulations to calculate or validate statistics predicted by the model. To generate simulation data, I used either GROMACS or LAMMPS simulation software to run parallelized MD simulations on a remote high performance computing platform. Processing simulation data required developing reusable and scalable code for data analysis in Python and Fortran. This work demanded a strong understanding of computational methods for modeling many-particle systems, applying these methods in a parallel computing environment, and contributing to a repository of software tools for data analysis and visualization. \\

At Thermo Fisher Scientific, I am currently working with the Focused Ion Beam (FIB) research group using computational methods to study novel FIB technologies. I have worked on several projects at Thermo Fisher, the largest of which is a computational modeling study to optimize FIB column elements for a particular column design. This project has involved running charged particle simulations in General Particle Tracer (GPT) on a local high performance computing cluster, and performing analyses on the simulation data which yield experimentally verifiable performance predictions. Performing accurate and realistic simulations in GPT requires an understanding of the boundary element method for computing electromagnetic fields, and writing models of FIB column elements in C to create the simulation environment. The simulation and data analysis workflow required for this project has been optimized using a proprietary software package written in Python, to which I have also made permanent contributions. \\

While my current experience is primarily in using computational models for data analysis, I am most interested in designing these models and developing software to implement them. I would love to be considered for a position at Flow Science. Thank you for your time. \\

Sincerely, \\
Jeremy Welsh-Kavan






\end{document}





%\begin{equation}
%\begin{split}
%\end{split}
%\end{equation}
