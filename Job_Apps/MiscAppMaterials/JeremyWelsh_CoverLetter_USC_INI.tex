
\documentclass[]{letter}
\usepackage[none]{hyphenat}
\usepackage[hmarginratio=1:1]{geometry}
\begin{document}

%{\Large Cover Letter}\\
%\end{center}
%\vspace{0.2 cm}

Dear Dr. Salminen, \\

I am excited to be applying for this position because it represents a great opportunity to expand and hone my current skillset as a computational scientist. My greatest intellectual interest is in computational science and I have been lucky enough to leverage this interest into a career path. My experience with computational science lies in two primary domains: molecular simulations of complex fluids, and simulations of charged particle optical systems. \\

As a graduate student at the University of Oregon, my research focused on using computational methods to build models of individual macromolecules or ensembles of macromolecules using a Langevin equation or the Ornstein–Zernike equation, respectively. In general, this required developing a theoretical model and using data from molecular dynamics (MD) simulations to calculate or validate statistics predicted by the model. To generate simulation data, I used either GROMACS or LAMMPS simulation software to run parallelized MD simulations on a remote high performance computing platform. Processing simulation data required developing reusable and scalable code for data analysis in Python and Fortran. This work demanded a strong understanding of computational methods for modeling many-particle systems, applying these methods in a parallel computing environment, and contributing to a repository of software tools for data analysis and visualization. \\

At Thermo Fisher Scientific, I worked with the Focused Ion Beam (FIB) research group using computational methods to study novel FIB technologies. I worked on several projects at Thermo Fisher, the largest of which is a computational modeling study to optimize FIB column elements for a particular column design. This project involved running charged particle simulations in General Particle Tracer (GPT) on a local high performance computing cluster, and performing analyses on the simulation data which yield experimentally verifiable performance predictions. Performing accurate and realistic simulations in GPT requires an understanding of the boundary element method for computing electromagnetic fields, and writing models of FIB column elements in C to create the simulation environment. The simulation and data analysis workflow required for this project was optimized using a proprietary software package written in Python, to which I have also made permanent contributions. \\


While my primary work experience is in developing models and data analysis techniques for physics, I am also deeply interested in applications of computational techniques to neuroscience and medicine. In addition to the experience described above, I have significant academic experience with machine learning and data science. I believe my scientific skillset, work experience, and interests are particularly well suited to the demands of this position. Thank you for your time. I look forward to hearing from you. \\

Sincerely, \\
Jeremy Welsh-Kavan






\end{document}





%\begin{equation}
%\begin{split}
%\end{split}
%\end{equation}
