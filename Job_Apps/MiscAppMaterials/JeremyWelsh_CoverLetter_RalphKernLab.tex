
\documentclass[]{letter}
\usepackage{amsmath,amssymb,amsthm}
\renewcommand{\qedsymbol}{$\blacksquare$}
\usepackage{amsmath}
\usepackage{bm}
\usepackage{amsfonts}
\usepackage{mathrsfs}
\usepackage{amssymb}
\usepackage{enumerate}
\usepackage{mdwlist}
\usepackage{dirtytalk}
\usepackage{xparse}
\usepackage{accents}
\usepackage[none]{hyphenat}
\usepackage[hmarginratio=1:1]{geometry}
\begin{document}

%{\Large Cover Letter}\\
%\end{center}
%\vspace{0.2 cm}

Dear Professor Kern and Professor Ralph, \\

I am very excited about this position because I have been interested in studying population genetics since I first learned about current research in computational biology and bioinformatics. I discovered your lab in my senior year of undergrad at University of Oregon and I have been interested in your work since then. I am currently searching for jobs doing scientific software development and I would be thrilled to do this kind of work in your lab. My relevant experience to this position consists primarily in writing scientific software for data generation, data analysis, and machine learning. 

As a graduate student at the University of Oregon, my research focused on using computational methods to build models of individual macromolecules or ensembles of macromolecules using a Langevin equation or the Ornstein–Zernike equation, respectively. In general, this required developing a theoretical model and using data from molecular dynamics (MD) simulations to calculate or validate statistics predicted by the model. To generate simulation data, I used either GROMACS or LAMMPS simulation software to run parallelized MD simulations on a remote high performance computing platform. Processing simulation data required developing reusable and scalable code for data analysis in Python and Fortran. This work demanded a strong understanding of computational methods for modeling many-particle systems, applying these methods in a parallel computing environment, and contributing to a repository of software tools for data analysis and visualization.

At Thermo Fisher Scientific, I am currently working with the Focused Ion Beam (FIB) research group using computational methods to study novel FIB technologies. I have worked on several projects at Thermo Fisher, the largest of which is a computational modeling study to optimize FIB column elements for a particular column design. This project has involved running charged particle simulations in General Particle Tracer (GPT) on a local high performance computing cluster, and performing analyses on the simulation data which yield experimentally verifiable performance predictions. Performing accurate and realistic simulations in GPT requires an understanding of the boundary element method for computing electromagnetic fields, and writing models of FIB column elements in C to create the simulation environment. The simulation and data analysis workflow required for this project has been optimized using a proprietary software package written in Python, to which I have also made substantial permanent contributions. 

In addition to my professional experience, I also have a personal interest in many topics which overlap with research in your lab. Specifically, I have a strong interest in the geometry of high-dimensional datasets and techniques for dimensionality reduction on these data, as exemplified in \textit{Visualizing Population Structure with Variational Autoencoders} (C. J. Battey, G. Coffing, A. Kern). I am also interested in more explicit models of population genetics such as the model described by the Fisher–KPP equation in \textit{Parallel adaptation: One or many waves of advance of an
advantageous allele?} (P. Ralph, G. Coop). 

My primary professional interest is in leveraging computational tools to do scientific research; and I have been interested in your research specifically for several years. I'm very excited about opportunities for this type of work and I look forward to speaking with you about this position. \\

Sincerely, \\
Jeremy Welsh-Kavan




\end{document}





%\begin{equation}
%\begin{split}
%\end{split}
%\end{equation}
