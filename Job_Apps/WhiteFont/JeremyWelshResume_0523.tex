
\documentclass[letterpaper,10.8pt]{article}

\usepackage{latexsym}
\usepackage[empty]{fullpage}
\usepackage{titlesec}
\usepackage{marvosym}
\usepackage[usenames,dvipsnames]{color}
\usepackage{verbatim}
\usepackage{enumitem}
\usepackage[pdftex]{hyperref}
\usepackage{fancyhdr}


\pagestyle{fancy}
\fancyhf{} % clear all header and footer fields
\fancyfoot{}
\renewcommand{\headrulewidth}{0pt}
\renewcommand{\footrulewidth}{0pt}

% Adjust margins
\addtolength{\oddsidemargin}{-0.375in}
\addtolength{\evensidemargin}{-0.375in}
\addtolength{\textwidth}{1in}
\addtolength{\topmargin}{-.5in}
\addtolength{\textheight}{1in}

\urlstyle{rm}

\raggedbottom
\raggedright
\setlength{\tabcolsep}{0in}

% Sections formatting
\titleformat{\section}{
  \vspace{-3pt}\scshape\raggedright\large
}{}{0em}{}[\color{black}\titlerule \vspace{-5pt}]

%-------------------------
% Custom commands
\newcommand{\resumeItem}[2]{
  \item\small{
    \textbf{#1}{: #2 \vspace{-2pt}}
  }
}

\newcommand{\resumeItemWithoutTitle}[1]{
  \item\small{
    {\vspace{-2pt}}
  }
}

\newcommand{\resumeSubheading}[4]{
  \vspace{-1pt} \item
    \begin{tabular*}{0.97\textwidth}{l@{\extracolsep{\fill}}r}
      \textbf{#1} & #2 \\
      #3  &  #4 \\
    \end{tabular*}\vspace{-5pt}
}


\newcommand{\resumeSubItem}[2]{\resumeItem{#1}{#2}\vspace{-4pt}}

\renewcommand{\labelitemii}{$\circ$}

\newcommand{\resumeSubHeadingListStart}{\begin{itemize}[leftmargin=*]}
\newcommand{\resumeSubHeadingListEnd}{\end{itemize}}
\newcommand{\resumeItemListStart}{\begin{itemize}[label={$\diamond$}]}
\newcommand{\resumeItemListEnd}{\end{itemize}\vspace{-5pt}}

%-------------------------------------------
%%%%%%  CV STARTS HERE  %%%%%%%%%%%%%%%%%%%%%%%%%%%%


\begin{document}

%----------HEADING-----------------
\begin{tabular*}{\textwidth}{l@{\extracolsep{\fill}}r}
  \textbf{{\LARGE Jeremy Welsh}}\\ 
Email : \href{mailto:jeremy@micromelody.net}{jeremy@micromelody.net} & Linkedin: \href{http://www.linkedin.com/in/jeremy-welsh}{www.linkedin.com/in/jeremy-welsh}  \\
Mobile : +1 (503) 890-1543  & Github: \href{https://github.com/jeremyiwk}{github.com/jeremyiwk}
\end{tabular*}

%--------PROGRAMMING SKILLS------------
\section{Technical Skills}
	\resumeSubHeadingListStart
	\resumeSubItem{Programming Languages}{Python, Fortran, C++, C, R, SQL, MATLAB, Julia, Shell
scripting (Unix/macOS), Mathematica}
	\resumeSubItem{Frameworks}{NumPy, Pandas, SciPy, Scikit-Learn, TensorFlow, OpenCV, PyMC, Matplotlib, Numba, ggplot2}
	\resumeSubItem{Software \& Tools}{Git, Docker, General Particle Tracer (GPT), GROMACS, LAMMPS, PyMol, \textcolor{white}{easily}}
	
\resumeSubHeadingListEnd 

%


%-----------EXPERIENCE-----------------
\section{Work Experience}
  \resumeSubHeadingListStart
    \resumeSubheading
    {Senior Intern}{Jun 2022 – Present}{Thermo Fisher Scientific}{Hillsboro, OR}
    
    \begin{itemize}[label={$\diamond$}]
    \itemsep0em %{description} [font=$\bullet$]
    	\item {Wrote Python scripts to automate milling and imaging procedures on dual-beam scanning electron microscope systems.}
	
	\item {Performed image registration on ion beam images using cross-correlation and a TensorFlow convolutional neural network to detect $\sim$100nm machining tolerances in ion column components.}
	
	\item {Developed Python code to measure optical aberrations in ion beam images using computer vision tools from OpenCV and Skimage.}
	
	\item {Wrote custom ion column elements in C for ion column simulations using GPT simulation software.}
	
	\item {Contributed to scientific software (Python) for parallelizing GPT simulations and processing data in a Linux HPC environment.}
	
	\item {Developed algorithms for regression of sparse simulation data to optimize novel ion column designs, resulting in up to 300\% improvement in ion beam performance.}
	
	\item {Used Python libraries such as NumPy, SciPy, Pandas, Matplotlib, and Seaborn for data analysis, visualization, and presentation to a team of scientists in order to direct critical decisions about experimental design.}
	
	\end{itemize} %{description}
      
    \resumeSubheading
		{Graduate Research Assistant}{Sep 2020 – Jun 2022}
		{University of Oregon}{Eugene, OR}
	\begin{itemize}[label={$\diamond$}]
	\itemsep0em  %{description} [font=$\bullet$]
	
	\item{Developed theoretical models for organic macromolecules at multiple
resolutions using mathematical tools from non-equilibrium statistical mechanics.}

	\item{Validated theoretical models against experimental data using molecular dynamics and Monte Carlo simulation data.}
	
		\item{Mentored undergraduate and graduate research assistants on projects related
to molecular coarse-graining schemes and simulation data analysis}
	
	\item{Performed and analyzed molecular dynamics simulations using GROMACS and LAMMPS molecular dynamics software on HPC clusters at San Diego Supercomputer Center.}
	
	\item{Characterized performance and the degree of parallelism of molecular dynamics simulations to determine computational resources requirement on 128 Core/node HPC system.}

	\item{Developed programs in Python and Fortran for data analysis of $\sim$10TB of molecular dynamics simulation data.}
	
	\item{Developed Fortran code to create input data for polymers of arbitrary length for MCCCS Towhee Monte Carlo molecular simulation software}

	\item{Developed novel coarse-grained models of DNA and validated models using custom Python and Fortran code, leading to improved agreement between predicted and simulated correlation statistics over prior models.}
	
	\item{Performed DBSCAN clustering on molecular dynamics simulation data to define regions for Markov Chain Monte Carlo simulation.}
	
	\item{Validated coarse-grained molecular models against predictions of statistical models such as principal component analysis (PCA) and time-lagged independent component analysis (t-ICA).}
	
	\item{Developed coarse-grained molecular model using PCA and an Autoencoder neural network to perform non-linear dimensionality reduction on the model parameter space.}
	
	\end{itemize} %{description}


    
\resumeSubHeadingListEnd

%-----------EDUCATION-----------------
\section{Education}
  \resumeSubHeadingListStart
    \resumeSubheading
      {University of Oregon}{Eugene, OR}
      {Master of Science, Physics, GPA: 3.92}{Sep 2020 - Jun 2022}
      
%	   {\scriptsize \textit{Courses: Operating Systems, Analysis Of Algorithms, Artificial %Intelligence, Machine Learning, Probability and Statistics and Network Security.}}
	   
    \resumeSubheading
      {University of Oregon}{Eugene, OR}
      {Bachelor of Science, Mathematics and Physics, GPA: 3.83 }{Sep 2016 - June 2020}
  \resumeSubHeadingListEnd



%-----------PROJECTS-----------------
%\section{Academic Projects}


%-----------Awards-----------------
%\section{Awards and Honors}
%\begin{description}[font=$\bullet$]
%\item {Departmental Honors in Physics} 
%\item {Latin Honors, Cum Laude}
%\item {Phi Beta Kappa Honors Society}
%\end{description}
%-------------------------------------------
\end{document}